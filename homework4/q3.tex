\documentclass[12pt, a4paper]{article}
\setlength{\parindent}{0pt}
\usepackage{amsmath}
\usepackage{amsthm}
\usepackage{amssymb}
\usepackage[a4paper, portrait, margin=1in]{geometry}

% i like the black square better.
\renewcommand{\qedsymbol}{$\blacksquare$}
%\renewcommand{\implies}{\Rightarrow}
\newcommand{\dispst}{\displaystyle}

\newcommand{\R}{\mathbb{R}}
\newcommand{\ddx}{\frac{d}{dx}}

\newtheorem{theorem}{Theorem}

% let's begin
\begin{document}

\textbf{(Q3)} 
\begin{proof}
    We know by the Intermediate Value Theorem (IVT) that if a function $f$:
    \begin{itemize}
        \item is continuous, and
        \item there exists $a,b, y\in \R$ where $f(a) < f(b)$ and $y \in (f(a), f(b))$,
    \end{itemize}

    then there exists $c \in \R$ where $f(c) = y$.

    Consdering the given equation $x^3 - x \cos x = 10$, we define it as a function
    $g(x) = x^3 - x \cos x - 10$, and we aim to show that $\exists c \in \R$ such that $f(c) = 0$.

    Since $x^3$, $x \cos x$ and $-10$ are all continuous, $g(x)$ is continuous by limit laws.

    Considering $g(2)$ and $g(2.5)$:
    \begin{align*}
        g(2) & = 8 - 2 \cos 2 - 10\\
        & = -2 - 2 \cos 2\\
        g(2.5) & = 15.625 - 2.5 \cos 2.5 - 10\\
        & = 5.625 - 2.5 \cos 2.5
    \end{align*}

    Since $\cos$ is bounded within $[-1, 1]$, $2 \cos 2 \in (-2, 2)$, and thus $-2 -2 \cos 2 < 0$.
    On the other hand, $2.5 \cos 2.5 < 0$ since $2.5 > \frac{\pi}{2}$, so $5.625 - 2.5 \cos 2.5 > 0$.

    Since $g(x)$ is continuous and $g(2) < 0$ while $g(2.5) > 0$, by IVT we can conclude that
    $\exists c \in (2, 2.5) \subseteq \R$ such that $f(c) = 0$; in other words, that $g(x)$ has a solution.
\end{proof}

Since the solution is within the interval $(2, 2.5)$, the closest integer is 2.



\end{document}