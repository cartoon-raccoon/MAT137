\documentclass[12pt, a4paper]{article}
\setlength{\parindent}{0pt}
\usepackage{amsmath}
\usepackage{amsthm}
\usepackage{amssymb}
\usepackage[a4paper, portrait, margin=1in]{geometry}

% i like the black square better.
\renewcommand{\qedsymbol}{$\blacksquare$}
\renewcommand{\epsilon}{\varepsilon}
\newcommand{\dispst}{\displaystyle}

\newcommand{\displim}[1]{\displaystyle{\lim_{#1}}}
\newcommand{\zerolim}{\displim{x \to 0}}
\newcommand{\st}{\text{ s.t. }}

\newcommand{\R}{\mathbb{R}}
\newcommand{\ddx}{\frac{d}{dx}}

\newtheorem{theorem}{Theorem}

% let's begin
\begin{document}

\textbf{(Q5)}

\begin{theorem}
    Prove that if $f$ is continuous at $x = a$, and $f(a) > 0$, then there exists
    a $\delta > 0$, so that $f(x) > 0$ for all $x \in (a - \delta, a + \delta)$.
\end{theorem}

\begin{proof}
    Formally speaking, the implication we are trying to prove is:
    \[
        \displim{x \to a} f(x) = f(a) > 0 \implies \exists \delta > 0 \st \forall
        x \in (a - \delta, a + \delta), \: f(x) > 0
    \]

    Assuming $f(x)$ is continuous by the limit definition of continuity (rewritten slightly):
    \begin{gather*}
        \forall \epsilon > 0, \: \exists \delta > 0 \st \forall x \in \R,\\
        x \in (a - \delta, a + \delta) \implies |f(x) - f(a)| < \epsilon
    \end{gather*}

    We know that there exists a $\delta$ for every $\epsilon$ such that the above implication
    holds. Therefore, all we need to do is to choose an $\epsilon$ such that $\epsilon < f(a)$,

    Let $\epsilon = \frac{f(a)}{2}$. Since $f(a) > 0$, it follows
    $f(a) - \frac{f(a)}{2} = \frac{f(a)}{2} > 0$. By the above implication, there exists
    a $\delta$ such that $|f(x) - f(a)| < \frac{f(a)}{2}$ holds for all $x \in (a - \delta, a + \delta)$.

    It follows:
    \[
        \forall x \in (a - \delta, a + \delta), \:|f(x) - f(a)| < \frac{f(a)}{2} \implies f(x) > 0
    \]

    Which proves the theorem, as required.
\end{proof}

\end{document}