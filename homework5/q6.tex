\documentclass[12pt, a4paper]{article}
\setlength{\parindent}{0pt}
\usepackage{amsmath}
\usepackage{amsthm}
\usepackage{amssymb}
\usepackage[a4paper, portrait, margin=1in]{geometry}

% i like the black square better.
\renewcommand{\qedsymbol}{$\blacksquare$}
%\renewcommand{\implies}{\Rightarrow}

\newcommand{\N}{\mathbb{N}}
\newcommand{\R}{\mathbb{R}}
\newcommand{\ddx}{\frac{d}{dx}}
\newcommand{\st}{\text{ s.t. }}
\newcommand{\fn}[3]{#1 \colon #2 \to #3}

\newtheorem{theorem}{Theorem}

% let's begin
\begin{document}

\textbf{(Q6)}

\begin{theorem}
    $|\sin x| \leq |x|$ for all $x \neq 0$.
\end{theorem}

\begin{proof}
    Fix $x$ as arbitrary.
    
    \textbf{Case 1: $x > 0$.} 
    We define a closed interval $[0, x]$. By MVT applied on $\sin x$:

    \begin{align*}
        \forall x > 0,\; \exists c \in (0, x) & \st \frac{\sin x - \sin 0}{x - 0} = \cos c\\
        & \implies \frac{\sin x}{x} = \cos c\\
    \end{align*}

    \textbf{Case 2: $x < 0$.}
    We define a closed interval $[x, 0]$. By MVT applied on $\sin x$:

    \begin{align*}
        \forall x < 0,\; \exists c \in (x, 0) & \st \frac{\sin 0- \sin x}{0 - x} = \cos c\\
        & \implies \frac{-\sin x}{-x} = \cos c\\
        & \implies \frac{\sin x}{x} = \cos c
    \end{align*}

    In both cases, we have $\displaystyle\frac{\sin x}{x} = \cos c$. Then,

    \begin{align*}
        \frac{\sin x}{x} = \cos c & \implies \left|\frac{\sin x}{x}\right| = |\cos c| \leq 1\\
        & \implies |\sin x| = |\cos c||x| \leq |x|
    \end{align*}
\end{proof}

\end{document}