\documentclass[12pt, a4paper]{article}
\setlength{\parindent}{0pt}
\usepackage{amsmath}
\usepackage{amsthm}
\usepackage{amssymb}
\usepackage[a4paper, portrait, margin=1in]{geometry}

% i like the black square better.
\renewcommand{\qedsymbol}{$\blacksquare$}
%\renewcommand{\implies}{\Rightarrow}

\newcommand{\N}{\mathbb{N}}
\newcommand{\R}{\mathbb{R}}
\newcommand{\ddx}{\frac{d}{dx}}
\newcommand{\st}{\text{ s.t. }}
\newcommand{\fn}[3]{#1 \colon #2 \to #3}

\newtheorem{theorem}{Theorem}

% let's begin
\begin{document}

\textbf{(Q4)}

\begin{proof}
    By earlier proof, $f(x)$ is differentiable. Thus,

    \[
        f'(x) = 12x^3 + 12x^2 + 12x = 12x(x^2 + x + 1)
    \]

    Since the discriminant of $x^2 + x + 1$ is less than 0, this
    expression has no real roots, thus the only root of $f'(x)$ is $x = 0$.

    By Rolle's Theorem, this means that $f(x)$ has at most 2 roots.

    We also have:

    \begin{align*}
        & f(-2) = 3(2)^4 + 4(2)^3 + 6(2)^2 - 10 = 94\\
        & f(0) = -10\\
        & f(1) = 3\\
    \end{align*}

    By the IVT, $\exists c \in (-2, 0) \st f(c) = 0$, and the same for $(0, 1)$.

    Thus, $f$ has two zeroes.
\end{proof}

\end{document}