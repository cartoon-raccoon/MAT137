\documentclass[12pt, a4paper]{article}
\setlength{\parindent}{0pt}
\usepackage{amsmath}
\usepackage{amsthm}
\usepackage{amssymb}
\usepackage[a4paper, portrait, margin=1in]{geometry}

% i like the black square better.
\renewcommand{\qedsymbol}{$\blacksquare$}
%\renewcommand{\implies}{\Rightarrow}

\newcommand{\N}{\mathbb{N}}
\newcommand{\R}{\mathbb{R}}
\newcommand{\ddx}{\frac{d}{dx}}
\newcommand{\st}{\text{ s.t. }}
\newcommand{\fn}[3]{#1 \colon #2 \to #3}

\newtheorem{theorem}{Theorem}

% let's begin
\begin{document}

\textbf{(Q2)}

\textit{(a)(i)}

\begin{proof}
    Since $f \circ g$ is invertible, it must be bijective and thus injective and
    surjective.

    By the definition of surjectivity of $f \circ g$:

    \[
        \forall y \in Z, \exists x \in X \st f(g(x)) = y
    \]

    Since by this definition, $f$ maps from every member in $g(X)$ to its codomain
    $Z$, it must be surjective.
\end{proof}

\textit{(a)(ii)}

\begin{proof}
    By earlier definition, $f \circ g$ is injective and surjective.

    Aiming for a contradiction, suppose that $g$ is not injective. Thus:
    
    \[
        \exists x_1, x_2 \in X \st g(x_1) = g(x_2) \text{ and } x_1 \neq x_2
    \]

    Then $g(x_1) = g(x_2) \implies f(g(x_1)) = f(g(x_2))$. However, simultaneously,
    $x_1 \neq x_2$. Thus, we have
    
    \[
        f(g(x_1)) = f(g(x_2)) \text{ and } x_1 \neq x_2
    \]

    Which suggests $f \circ g$ is not injective.

    This is a contradiction, and thus $g$ must be injective.
\end{proof}

\textit{(b)}

Let $\fn{g}{\R}{\R}$ be given by $g(x) = e^x$, and $\fn{f}{\R}{(0, \infty)}$ be
given by $f(x) = x^2$. Thus, $\fn{f \circ g}{\R}{(0, \infty)}$ is given by
$f(g(x)) = (e^x)^2 = e^{2x}$, which is bijective and is therefore invertible.
\end{document}