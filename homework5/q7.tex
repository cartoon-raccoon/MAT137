\documentclass[12pt, a4paper]{article}
\setlength{\parindent}{0pt}
\usepackage{amsmath}
\usepackage{amsthm}
\usepackage{amssymb}
\usepackage[a4paper, portrait, margin=1in]{geometry}

% i like the black square better.
\renewcommand{\qedsymbol}{$\blacksquare$}
%\renewcommand{\implies}{\Rightarrow}

\newcommand{\N}{\mathbb{N}}
\newcommand{\R}{\mathbb{R}}
\newcommand{\ddx}{\frac{d}{dx}}
\newcommand{\st}{\text{ s.t. }}
\newcommand{\fn}[3]{#1 \colon #2 \to #3}

\newtheorem{theorem}{Theorem}

% let's begin
\begin{document}

\textbf{(Q7)}

\textit{(a)} True.

\begin{proof}
    Since $f$ is strictly increasing, we have
    \[
        \forall x_1, x_2 \in \R, x_1 < x_2 \implies f(x_1) < f(x_2)
    \]

    For the sake of contradiction, assume $f$ has a maximum or minimum. By
    definition of a local minimum or maximum, we have that $x_1 \in \R$ is a
    maximum iff

    \[
        \exists \delta > 0 \st x_2 \in (x_1 - \delta, x_1 + \delta) 
        \implies f(x_2) \leq f(x_1)
    \]

    Let $x_1 = c$ and $x_2 = x_1 + \frac{\delta}{2}$. We have $x_1 < x_2$, so
    $f(x_1) < f(x_2)$, which contradicts the assumption that $f$ has a maximum.

    This argument also holds for a minimum.

\end{proof}

\textit{(b)} False.

Let $\fn{f}{\R}{\R}$ be given by $f(x) = x^3$, which is strictly increasing
and also differentiable. $f'(x) = 3x^2$, thus $f'(0) = 0$.

\textit{(c)} True.

The proof follows from proof of Q2. If $f \circ g$ is bijective, then $f$ must
be surjective and $g$ must be injective. Thus if $f \circ f$ is bijective, then
$f$ must be both surjective and injective, and thus bijective.

\textit{(d)} True.

\begin{proof}
    By the definition of continuity, we have
    \begin{gather*}
        \forall \varepsilon > 0, \exists \delta > 0 \st \forall x \in \R,\\
        |x - p| < \delta \implies |f'(x) - f'(p)| < \varepsilon
    \end{gather*}

    Let $\varepsilon = \displaystyle\frac{f'(p)}{2}$.
    Since $f'$ is continuous, we know that such a $\delta$ exists to
    satisfy this implication. Let $I$ be $(p - \delta, p + \delta)$.
    Since $f'(p) > 0$,

    \[
        |x - p| < \delta \implies |f'(x) - f'(p)| < \frac{f'(p)}{2}
        \implies f'(x) > 0
    \]

    Since $\forall x \in I,\; f'(x) > 0$, $f$ is strictly increasing,
    and thus injective.
\end{proof}

\end{document}