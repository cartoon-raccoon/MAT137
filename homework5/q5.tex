\documentclass[12pt, a4paper]{article}
\setlength{\parindent}{0pt}
\usepackage{amsmath}
\usepackage{amsthm}
\usepackage{amssymb}
\usepackage[a4paper, portrait, margin=1in]{geometry}

% i like the black square better.
\renewcommand{\qedsymbol}{$\blacksquare$}
%\renewcommand{\implies}{\Rightarrow}

\newcommand{\N}{\mathbb{N}}
\newcommand{\R}{\mathbb{R}}
\newcommand{\ddx}{\frac{d}{dx}}
\newcommand{\st}{\text{ s.t. }}
\newcommand{\fn}[3]{#1 \colon #2 \to #3}

\newtheorem{theorem}{Theorem}

% let's begin
\begin{document}

\textbf{(Q5)}

\textit{(a)}

Since $f$ is defined on a closed interval, this follows from the Extreme Value Theorem.

\textit{(b)}

We know that maximum and minimum values occur when $f'(x) = 0$. Therefore:

\[
    f'(x) = \frac{7}{3}x^{\frac{4}{3}} - \frac{7}{3}x^{-\frac{2}{3}}\\
\]
\begin{align*}
    f'(x) = 0 & \implies \frac{7}{3}x^{\frac{4}{3}} = \frac{7}{3}x^{-\frac{2}{3}}\\
    & \implies x^{\frac{4}{3}} = x^{-\frac{2}{3}}\\
    & \implies x^{\frac{4}{3}} = \frac{1}{x^{\frac{2}{3}}}\\
    & \implies x^{\frac{6}{3}} = x^2 = 1\\
    & \implies x = \pm 1
\end{align*}

Therefore, $f$ has maximum and/or minimum points at $-1$ and 1. Then,

\[
    f''(x) = \frac{29}{9}x^{\frac{1}{3}} - \frac{14}{9}x^{-\frac{5}{3}}
\]

$f''(-1) < 0$, so $f$ is concave down at that point and thus $x = -1$
is a maximum point, and $f''(1) > 0$, so $f$ is concave up at that point,
and thus $x = 1$ is a minimum point.

\end{document}