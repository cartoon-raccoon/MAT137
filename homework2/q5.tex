\documentclass[12pt, a4paper]{article}
\setlength{\parindent}{0pt}
\usepackage{amsmath}
\usepackage{amssymb}
\usepackage{amsthm}
\usepackage[a4paper, portrait, margin=1in]{geometry}

\renewcommand{\qedsymbol}{$\blacksquare$}
\renewcommand{\epsilon}{\varepsilon}

\newcommand{\displim}[1]{\displaystyle{\lim_{#1}}}
\newcommand{\zerolim}{\displim{x \to 0}}
\newcommand{\st}{\text{ s.t. }}

%\renewcommand{\implies}{\Rightarrow}
\newcommand{\R}{\mathbb{R}}

\newtheorem{theorem}{Theorem}

% let's begin
\begin{document}
\textbf{(Q5)}

\textit{(a)}
The negation of the definition of limit is:
\[
    \exists \epsilon > 0 \text{ s.t. } \forall \delta > 0, \exists x \in \R
    \text{ s.t. } 0 < |x - a| < \delta \text{ and } |f(x) - L| \geq \epsilon
\]

\textit{(b)}
\begin{gather*}
    \displim{x \to a} f(x) \text{ DNE iff}\\
    \forall L \in \R, \exists \epsilon > 0 \text{ s.t. }\forall \delta > 0, \exists
    x \in \R \text{ s.t. }\\
    0 < |x - a| < d \text{ and } |f(x) - L| \geq \epsilon
\end{gather*}

\textit{(c)}
\begin{proof}
    Formally speaking, the implication we are looking to prove is:
    \begin{gather*}
        \forall L \in \R\: \exists \epsilon_1 \st \forall \delta >0, \exists
        x \in \R \st 0 < |x - a| < \delta \text{ and } |f(x) - L| \geq \epsilon_1\\
        \iff\\
        \forall L \in \R\: \exists \epsilon \st \forall \delta >0, \exists
        x \in \R \st 0 < |x - a| < \delta \text{ and } |cf(x) - cL| \geq \epsilon\\
    \end{gather*}

    Assuming the first implication, we can assume theat there exists a $\epsilon_1$
    that satisfies $|f(x) - L \geq \epsilon_1$. We need to find an $\epsilon$ that
    satisfies $|cf(x) -cL \geq \epsilon$.

    Fix $\delta > 0$. Let $\epsilon = \frac{\epsilon_1}{c}$.

    We observe that $|cf(x) - cL| = c|f(x)- L|$.

    Then,
    \begin{gather*}
        c|f(x)- L| \geq \frac{\epsilon_1}{c} \implies |f(x) -L| \geq \epsilon_1
    \end{gather*}

    Now, assuming the second implication, we observe that 
    $\displaystyle{\frac{\epsilon}{c}} < \epsilon_1$. Thus,
    \[
        |f(x) - L| \geq \epsilon_1 > \frac{\epsilon}{c}
    \]
\end{proof}

\textit{(d)}
\begin{proof}
    By contrapositive of the given implication:

    \begin{quote}
        $\displim{x \to a}\: cf(x)$ exists iff $\displim{x \to a}\: f(x)$ exists
    \end{quote}

    Considering one way of the implication, if $\displim{x \to a}\: f(x)$ exists
    and $\displim{x \to a}\: c$ exists, by limit laws we can calculate
    $\displim{x \to a}\: cf(x)$ which also exists.

    Considering the other direction of the implication, if $\displim{x \to a}\: cf(x)$
    exists and $\displim{x \to a}\: c$ exists and is non-zero, then we can perform
    the following division:
    \[
        \frac{\displim{x \to a}\: cf(x)}{\displim{x \to a}\: c}
    \]
    Which would yield $\displim{x \to a}\: f(x)$.
\end{proof}

\textit{(e)}
Consider the function $\displaystyle{\frac{1}{x}}$.

$\displim{x \to o}\: \frac{1}{x} \text{ DNE}$, but 
$\displim{x \to 0}\: 0 \cdot \frac{1}{x} = 0$
\end{document}