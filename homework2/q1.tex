\documentclass[12pt, a4paper]{article}
\setlength{\parindent}{0pt}
\usepackage{amsmath}
\usepackage{amssymb}
\usepackage[a4paper, portrait, margin=1in]{geometry}

% let's begin
\begin{document}
\textbf{(Q1)}

\textit{(a)}
We can factorize $\displaystyle{\frac{x + 1}{x^2 - x}}$ as follows:
\[
    \frac{x + 1}{x(x - 1)}
\]

Considering the denominator, we can only have a negative value of the term
when $x > 0, x < 1$. Considering the numerator, the entire term stays negative
for all $x < -1$, as all $x$ terms will be negative.

Thus, the final interval is:
\[
    (-\infty, -1) \cup (0, 1)
\]

\textit{(b)}
We can factorise $3 < |3x + 9| < 12$:
\begin{gather*}
    3 < 3|x + 3| < 12\\
    1 < |x + 3| < 4
\end{gather*}

From this inequality we can infer
\begin{gather*}
    1 < |x + 3| + 4 \implies -4 < x + 3 < -1 \text{ and } 1 < x + 3 < 4\\
    \implies -7 < x < -4 \text{ and } -2 < x < 1
\end{gather*}

The final interval is:
\[
    (-7, -4) \cup (-2, 1)
\]

\end{document}
