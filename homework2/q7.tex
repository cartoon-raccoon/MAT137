\documentclass[12pt, a4paper]{article}
\setlength{\parindent}{0pt}
\usepackage{amsmath}
\usepackage{amsthm}
\usepackage{amssymb}
\usepackage[a4paper, portrait, margin=1in]{geometry}

% i like the black square better.
\renewcommand{\qedsymbol}{$\blacksquare$}
\renewcommand{\epsilon}{\varepsilon}

\newcommand{\displim}[1]{\displaystyle{\lim_{#1}}}
\newcommand{\zerolim}{\displim{x \to 0}}

\newcommand{\R}{\mathbb{R}}
\newcommand{\Z}{\mathbb{Z}}
%\renewcommand{\implies}{\Rightarrow}

\newtheorem{theorem}{Theorem}

% let's begin
\begin{document}

\textbf{(Q7)}

\textit{(a)} False.

Consider the piecewise function

\[
    f(x) = \begin{cases}
        \frac{\sin x}{x} + 2021 & \text{if } x \neq 0\\
        69 & \text{if } x = 0
    \end{cases}
\]

This function fulfills the following:

\begin{itemize}
    \item It is defined for all $x \in \R$.
    \item It is bounded within $(2021, 2022)$ for all $x \neq 0$, and $(0, 2022)$ for $x = 0$,
    which means the entire function is bounded within $(0, 2022)$.
    \item $\zerolim f(x) = 2022$.
\end{itemize}

As such, it is possible to have $0 < f(x) < 2022$ with a limit $L \geq 2022$.

\textit{(b)} False.

We cannot draw conclusions given only the limit of the product of
two functions, as we cannot guarantee that there exists a limit for either function.

As a counter-example, consider $f(x) = x, \: g(x) = \displaystyle{\frac{1}{x}}$. It follows:
\[
    \begin{gathered}
        \zerolim \: x \cdot \frac{1}{x} = 1\\
        \zerolim \: x = 0\\
        \zerolim \: \frac{1}{x} \text{ DNE}
    \end{gathered}
\]

Therefore, $\displim{x \to a} \: f(x) = \displaystyle{\frac{1}{\displim{x \to a} \: g(x)}}$
is not true for some functions.

\textit{(c)} True.

\begin{proof}
    We are given that
    \[
        \forall \epsilon \in (0, \tfrac{1}{100}),\: \exists \: \delta_1 > 0 \text{ s.t. }
        \forall x \in \R,\: 0 < |x - 137| < \delta_1 \implies |f(x) - 2022| < \epsilon
    \]

    Thus we can assume that for $\epsilon < \tfrac{1}{100}$, there is a $\delta_1$ for that
    $\epsilon$ that satisfies this particular implication. We want to show that:

    \[
        \forall \epsilon > 0,\: \exists \: \delta > 0 \text{ s.t. }
        \forall x \in \R,\: 0 < |x - 137| < \delta \implies |f(x) - 2022| < \epsilon
    \]

    Fix $\epsilon \geq \tfrac{1}{100}$. Let $\delta = \delta_1$. It follows that:

    \begin{gather*}
        0 < |x - 137| < \delta_1 = \delta \\
        \implies 0 < |x - 137| < \delta \implies |f(x) - 2022| < \tfrac{1}{100} \leq \epsilon\\
        \implies 0 < |x - 137| < \delta \implies |f(x) - 2022| < \epsilon
    \end{gather*}
\end{proof}

\end{document}