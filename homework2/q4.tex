\documentclass[12pt, a4paper]{article}
\setlength{\parindent}{0pt}
\usepackage{amsmath}
\usepackage{amsthm}
\usepackage{amssymb}
\usepackage[a4paper, portrait, margin=1in]{geometry}

% i like the black square better.
\renewcommand{\qedsymbol}{$\blacksquare$}
\renewcommand{\epsilon}{\varepsilon}

\newcommand{\R}{\mathbb{R}}
\newcommand{\Z}{\mathbb{Z}}
%\renewcommand{\implies}{\Rightarrow}

\newtheorem{theorem}{Theorem}

% let's begin
\begin{document}

\textbf{(Q4)}
\begin{theorem}
    Prove that $\displaystyle{\lim_{x \to 3} \frac{x^2 - 9}{x - 3}} = 6$ using the $\delta-\epsilon$ definition of the limit.
\end{theorem}

\begin{proof}
    We observe that 
    \[
        \frac{x^2 - 9}{x - 3} = \frac{(x + 3)(x - 3)}{(x - 3)} = x + 3
    \]
    Since we are working with limits, we can remove common factors with impunity as we do not
    risk dividing by 0.

    The $\delta-\epsilon$ definition of the limit for this function is
    \[
        \forall \epsilon > 0, \exists \delta > 0 \text { s.t. } \forall x \in \R, \:
        0 < |x - 3| < \delta \implies |(x + 3) - 6| < \epsilon
    \]

    We can simplify the right hand side of this implication to $|x - 3| < \epsilon$.

    Let $\delta = \epsilon$. It follows that:

    \begin{align*}
        0 < |x - 3| < \delta \implies 0 < |x - 3| < \epsilon \implies |x - 3| < \epsilon
    \end{align*}
\end{proof}

\end{document}