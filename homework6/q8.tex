\documentclass[12pt, a4paper]{article}
\setlength{\parindent}{0pt}
\usepackage{amsmath}
\usepackage{amsthm}
\usepackage{amssymb}
\usepackage[a4paper, portrait, margin=1in]{geometry}

% i like the black square better.
\renewcommand{\qedsymbol}{$\blacksquare$}
%\renewcommand{\implies}{\Rightarrow}

\newcommand{\N}{\mathbb{N}}
\newcommand{\Q}{\mathbb{Q}}

\newcommand{\ddx}{\frac{d}{dx}}
\newcommand{\st}{\; \colon \;}
\newcommand{\sse}{\subseteq}
\newcommand{\lsum}[2]{L_{#1}(#2)}
\newcommand{\usum}[2]{U_{#1}(#2)}

\newtheorem{theorem}{Theorem}

% let's begin
\begin{document}

\textbf{(Q8)}

\textit{(a)}
Let $x, y \in [a, b]$ and $x < y$.
We observe the following:

\begin{gather*}
    \exists \;t \in [x, y] \st t \in \Q\\
    \exists \;s \in [x, y] \st s \notin \Q
\end{gather*}

Therefore, $\inf{f}$ on any subinterval of $[a, b]$ on $f$ is always 2022.
In addition, $\sup{f}$ on any subinterval of $[a, b]$ on $f$ is always 2023.

It follows that for any partition $P$ of $[a,b]$:

\[
    \lsum{P}{f} = 2022(b - a), \quad \usum{P}{f} = 2023(b - a)
\]

\textit{(b)}

\begin{proof}
    Since $\lsum{P}{f} = 2022(b - a)$ and $\usum{P}{f} = 2023(b - a)$ for
    any partition $P$ of $[a, b]$ on $f$, we have:

    \begin{gather*}
        \overline{I}\;^{b}_{a} (f) = \inf\{ 2023(b - a) \} = 2023(b - a)\\
        \underline{I}\;^{b}_{a} (f) = \sup\{ 2022(b - a) \} = 2022(b - a)
    \end{gather*}

    Since $a < b$, $b - a > 0$ and thus $2023(b - a) > 2022(b - a)$.

    Therefore, the upper and lower integrals are not equal and $f$ is not
    integrable.
\end{proof}

\end{document}