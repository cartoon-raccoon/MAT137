\documentclass[12pt, a4paper]{article}
\setlength{\parindent}{0pt}
\usepackage{mathtools}
\usepackage{amsmath}
\usepackage{amsthm}
\usepackage{amssymb}
\usepackage[a4paper, portrait, margin=1in]{geometry}

% i like the black square better.
\renewcommand{\qedsymbol}{$\blacksquare$}
%\renewcommand{\implies}{\Rightarrow}

\newcommand{\N}{\mathbb{N}}
\newcommand{\ddx}{\frac{d}{dx}}
\newcommand{\displim}[1]{\displaystyle{\lim_{#1}}}
\newcommand{\inflim}{\displim{x \to \infty}\;}

\newcommand\lheq{\stackrel{\mathclap{\mbox{\tiny{L'H}}}}{=}}

\newtheorem{theorem}{Theorem}

% let's begin
\begin{document}

\textbf{(Q3)}

We can express $\left(1 + \frac{M}{x}\right)^{Nx}$ as $e^{Nx \cdot \ln(1 + \frac{M}{x})}$.

Then: \begin{align*}
    \inflim Nx \cdot \ln(1 + \frac{M}{x}) & = N \cdot \frac{\ln\left(1 + \frac{M}{x}\right)}{\frac{1}{x}}\\
    & \lheq \inflim N \cdot
    \frac{\frac{1}{1 + \frac{M}{x}} \cdot \frac{-M}{x^2}}{-\frac{1}{x^2}}\\
    & = \inflim N \cdot \frac{M}{\frac{M + x}{x}}\\
    & = \inflim N \cdot M \cdot \frac{x}{x + M}\\
    & \lheq \inflim N \cdot M \cdot \frac{1}{1}\\
    & = \inflim M \cdot N
\end{align*}

Thus, $\inflim \left(1 + \frac{M}{x}\right)^{Nx} = e^{MN}$.

\end{document}