\documentclass[12pt, a4paper]{article}
\setlength{\parindent}{0pt}
\usepackage{amsmath}
\usepackage{amsthm}
\usepackage{amssymb}
\usepackage[a4paper, portrait, margin=1in]{geometry}

% i like the black square better.
\renewcommand{\qedsymbol}{$\blacksquare$}
\renewcommand{\epsilon}{\varepsilon}
%\renewcommand{\implies}{\Rightarrow}

\newcommand{\N}{\mathbb{N}}
\newcommand{\ddx}{\frac{d}{dx}}
\newcommand{\sse}{\subseteq}
\newcommand{\lsum}[2]{L_{#1}(#2)}
\newcommand{\usum}[2]{U_{#1}(#2)}

\newtheorem{theorem}{Theorem}

% let's begin
\begin{document}

\textbf{(Q5)}

\begin{theorem}
    $f(x) = x^2 - 1$ is integrable on $[-1, 1]$.
\end{theorem}

\begin{proof}
    We use the $\epsilon$-characterisation of integrability for this proof:

    \[
        \forall \epsilon > 0, \; \exists \text{ a partition $P$ of } [a,b] 
        \colon \;\; U_P(f) - L_P (f) < \epsilon
    \]

    We evaluate $U_P(f) - L_P(f)$ by cases for odd and even $n$.

    For even $n$:
    \[
        U_P(f) - L_P(f) = \frac{4}{n}
    \]

    For odd $n$:
    \[
        U_P(f) - L_P(f) = \frac{4}{n} - \frac{2}{n^3}
    \]

    Ultimately, we aim to prove:
    \[
        \forall \epsilon > 0, \exists n \in \N \colon \;\;
        U_{P_n}(f) - L_{P_n}(f)< \epsilon
    \]

    Fix $\epsilon > 0$. For even $n$, let $n$ be a natural such that $n > \frac{4}{\epsilon}$.
    Then \[
        \frac{4}{n} < \frac{4}{\frac{4}{\epsilon}} \implies
        \frac{4}{n} < 4 \cdot \frac{\epsilon}{4} \implies
        \frac{4}{n} < \epsilon
    \]

    For odd $n$, let $n$ be a natural such that 
    $n > \max\{\frac{8}{\epsilon}, \left(\frac{4}{\epsilon}\right)^{\frac{1}{3}}\}$. Then
    \[
        \frac{4}{n} - \frac{2}{n^3} < \frac{\epsilon}{2} + \frac{\epsilon}{2} = \epsilon
    \]

    Since we have proven that this implication holds for all $n$, $f$ is integrable on $[-1, 1]$.

\end{proof}

\end{document}