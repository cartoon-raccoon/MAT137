\documentclass[12pt, a4paper]{article}
\setlength{\parindent}{0pt}
\usepackage{amsmath}
\usepackage{amsthm}
\usepackage{amssymb}
\usepackage[a4paper, portrait, margin=1in]{geometry}

% i like the black square better.
\renewcommand{\qedsymbol}{$\blacksquare$}
%\renewcommand{\implies}{\Rightarrow}

\newcommand{\N}{\mathbb{N}}
\newcommand{\ddx}{\frac{d}{dx}}

\newtheorem{theorem}{Theorem}

% let's begin
\begin{document}

\textbf{(Q4)}

We observe that $f$ is symmetrical about $x = 0$, this we can take
advantage of this symmetry to simplify our expressions for upper bounds.

However, we need to consider separate cases for even $n$ and odd $n$.
As such, for even $n$:

\begin{align*}
    U_{P_n} & = \frac{2}{n}\left[2 \cdot \sum_{i = 1}^{\frac{n}{2}}
    f\left(\frac{2i}{n}\right)\right]\\
    \\
    L_{P_n} & = \frac{2}{n}\left[2 \cdot \sum_{i = 0}^{\frac{n}{2}}
    f\left(\frac{2i}{n}\right)\right]\\
\end{align*}

And for odd $n$:

\begin{align*}
    U_{P_n} & = \frac{2}{n}\left[
        f\left(\frac{1}{n}\right) + 2 \cdot 
        \sum_{i = 1}^{\frac{n - 1}{2}}
    f\left(\frac{2i + 1}{n}\right)\right]\\
    \\
    L_{P_n} & = \frac{2}{n}\left[
        f\left(\frac{1}{n}\right) + 2 \cdot 
        \sum_{i = 1}^{\frac{n - 1}{2}}
    f\left(\frac{2i - 1}{n}\right)\right]\\
\end{align*}

\end{document}