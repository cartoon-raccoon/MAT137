\documentclass[12pt, a4paper]{article}
\setlength{\parindent}{0pt}
\usepackage{amsmath}
\usepackage{amsthm}
\usepackage{amssymb}
\usepackage[a4paper, portrait, margin=1in]{geometry}

% i like the black square better.
\renewcommand{\qedsymbol}{$\blacksquare$}
%\renewcommand{\implies}{\Rightarrow}

\newcommand{\N}{\mathbb{N}}
\newcommand{\Q}{\mathbb{Q}}

\newcommand{\ddx}{\frac{d}{dx}}
\newcommand{\st}{\; \colon \;}
\newcommand{\sse}{\subseteq}
\newcommand{\lsum}[2]{L_{#1}(#2)}
\newcommand{\usum}[2]{U_{#1}(#2)}

\newtheorem{theorem}{Theorem}

% let's begin
\begin{document}

\textbf{(Q9)}

\textit{(a)} True.

\begin{proof}
    We observe that:

    \begin{align*}
        f + 2g & = 2g + f\\
        & = 2g - 2f + 3f\\
        & = 2(g - f) + 3f
    \end{align*}

    Since $f + 2g$ can be expressed as a combination and transformation
    of two integrable functions, it is also integrable.
\end{proof}

\textit{(b)} False.

As a counterexample, let $f = g = \chi_{\Q}$, where

\[
    \chi_{\Q}(x) = \begin{cases}
        1 \; \colon x \in \Q\\
        0 \; \colon x \notin \Q\\
    \end{cases}
\]

Neither $f$ nor $g$ are integrable, while $(f \circ g) (x) = 1$, which is integrable.

\textit{(c)} False.

Let $\chi_{\Q}$ be the same as the one defined in (b). We define $f$ as:

\[
    f(x) = \begin{cases}
        x^2 \; \colon x \in [-2 ,0] \cup [1, 6]\\
        \chi_{\Q} \; \colon x \in (0, 1)
    \end{cases}
\]

$f$ is integrable on $[-2, 0]$ and $[1, 6]$ but not on $(0, 1)$, so it is not
integrable on $[-1, 4]$.

\textit{(d)} False.

Let $f$ be defined as:

\[
    f(x) = \begin{cases}
        1 \; \colon x \in \Q\\
        -1 \; \colon x \notin \Q\\
    \end{cases}
\]

$f^2(x) = 1$, which is integrable, while $f$ is not.

\end{document}