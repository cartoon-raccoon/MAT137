\documentclass[12pt, a4paper]{article}
\setlength{\parindent}{0pt}
\usepackage{amsmath}
\usepackage{amsthm}
\usepackage{amssymb}
\usepackage[a4paper, portrait, margin=1in]{geometry}

\renewcommand{\qedsymbol}{$\blacksquare$}

\newcommand{\R}{\mathbb{R}}
\newcommand{\Z}{\mathbb{Z}}
\newcommand{\displim}[1]{\displaystyle{\lim_{#1}\:}}
\newcommand{\zerolim}{\displim{x \to 0}}

\newtheorem{theorem}{Theorem}

% let's begin
\begin{document}
\textbf{(Q4)}

\begin{theorem}
    Let $f \colon \R \to \R$ be a function so that
    \[
        (1 - \cos^2x) \leq f(x) \leq x^2
    \]
    for all $x \in (-2022, 2022)$.

    Prove that $\zerolim f(x)$ exists.
\end{theorem}

\begin{proof}
    We observe that $1 - \cos^2x = \sin^2x$. We can thus rewrite the definition of
    $f(x)$ as:

    \[
        \sin^2x \leq f(x) \leq x^2
    \]

    Using the fact that $\sin x$ is continuous, we can calculate $\displim{x \to 0} \sin x$
    by evaluating it as follows:
    \[
        \displim{x \to 0} \sin x = \sin 0 = 0
    \]
    
    From which we can use limit laws to compute $\zerolim \sin^2x$:

    \[
        \zerolim \sin x = 0 \implies \zerolim (\sin x)(\sin x) = 0 \cdot 0 = 0
    \]

    Using the fact that $x^2$ is continuous, we can also compute its limit at $x = 0$:

    \[
        \zerolim x^2 = 0^2 = 0
    \]

    By Squeeze Theorem,

    \begin{gather*}
        \forall x \in (-2022, 2022),\: \sin^2x \leq f(x) \leq x^2\\
        \text{ and }\\
        \zerolim \sin^2x = \zerolim x^2 = 0\\
        \implies \zerolim f(x) = 0
    \end{gather*}

    Which also proves that $\zerolim f(x)$ exists, as required.
\end{proof}

\end{document}