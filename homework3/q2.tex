\documentclass[12pt, a4paper]{article}
\setlength{\parindent}{0pt}
\usepackage{amsmath}
\usepackage{amssymb}
\usepackage[a4paper, portrait, margin=1in]{geometry}

\newcommand{\displim}[1]{\displaystyle{\lim_{#1}}}

% let's begin
\begin{document}
\textbf{(Q2)}

By limit laws, we know that:

\[
    \displim{x \to a}\: f(x) = L \text{ and }
    \displim{x \to a}\: g(x) = M \implies
    \displim{x \to a}\: f(x) + g(x) = L + M
\]
\[
    \displim{x \to a}\: f(x) = L \text{ and }
    \displim{x \to a}\: g(x) = M \implies
    \displim{x \to a}\: f(x) \cdot g(x) = L \cdot M
\]

Since $\displim{x \to \infty}\: \tfrac{1}{x} = 0$, we can safely remove any
operation involving adding a rational term.

\textit{(a)}

First, some manipulations are in order:

\begin{align*}
    \frac{
        2x^2 - 16x + 1
    }{
        x^3 + 137x + 2022
    } & = 
    \frac{
        x^2(2 - \frac{16}{x} + \frac{1}{x^2})
    }{
        x^2(x + \frac{137}{x} + \frac{2022}{x^2})
    }\\ & = 
    \frac{
        2 - \frac{16}{x} + \frac{1}{x^2}
    }{
        x + \frac{137}{x} + \frac{2022}{x^2}
    }
\end{align*}

Removing the rational terms we get:

\[
    \displim{x \to \infty}\:\frac{2 - \frac{16}{x} + \frac{1}{x^2}}{x + \frac{137}{x} + \frac{2022}{x^2}}
    = \displim{x \to \infty}\: \frac{2}{x}
    = \displim{x \to \infty}\: 2 \cdot \displim{x \to \infty}\: \frac{1}{x}
\]

By the limit laws,

\[
    \displim{x \to \infty}\: 2 \cdot \displim{x \to \infty}\: \frac{1}{x} = 2 \cdot 0 = 0
\]

\textit{(b)}

Similar manipulations are in order:

\begin{align*}
    \frac{
        137x^5 - 10000x + 12
    }{
        2022x^5 + 20x^4 - x + 1
    } & =
    \frac{
        x^5(137 - \frac{100000}{x^4} + \frac{12}{x^5})
    }{
        x^5(2022 + \frac{20}{x} - \frac{1}{x^4} + \frac{1}{x^5})
    }\\ & =
    \frac{
        137 - \frac{100000}{x^4} + \frac{12}{x^5}
    }{
        2022 + \frac{20}{x} - \frac{1}{x^4} + \frac{1}{x^5}
    }
\end{align*}

Removing the rational terms:

\[
    \displim{x \to \infty}\: \frac{
        137 - \frac{100000}{x^4} + \frac{12}{x^5}
    }{
        2022 + \frac{20}{x} - \frac{1}{x^4} + \frac{1}{x^5}
    } = \frac{137}{2022}
\]


\end{document}