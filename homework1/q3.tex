\documentclass[12pt, letterpaper]{article}
\setlength{\parindent}{0pt}
\usepackage{amsmath}
\usepackage{amsthm}
\usepackage{amssymb}

\begin{document}
    \textbf{(Q3)}

    We begin with a few definitions:

    \begin{itemize}
        \item Let $s$ be a fixed student taking MAT137.
        \item Let $M$ be the set of MAT137 students.
        \item Let $F$ be the set of friends of $s$.
        \item Let $N_{F}$ be a set of friends of $s$ with first name starting with P.
        \item Let $N_{L}$ be a set of friends of $s$ with last name starting with P.
    \end{itemize}

    Thus, the statement
    \begin{quote}
        There is a student taking MAT137 at UTM who has no friends whose first
        name and last name begin with the letter P.
    \end{quote}

    This can be expressed as:

    \[
        \exists s \in M \text{ s.t. } \forall p \in F, p \notin N_{F} \text{ and } p \notin N_{L}
    \]
    This negates to:
    \[
        \forall s \in M, \exists p \in F \text{ s.t. } p \in N_{F} \text{ or } p \in N_{L}
    \]
    Which can be expressed in plain English as:

    \begin{quote}
        All students taking MAT137 at UTM have at least one friend whose first or last name begins with P.
    \end{quote}
\end{document}