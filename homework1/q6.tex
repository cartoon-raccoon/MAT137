\documentclass[12pt, a4paper]{article}
\setlength{\parindent}{0pt}
\usepackage{amsmath}
\usepackage{amsthm}
\usepackage{amssymb}
\usepackage[a4paper, portrait, margin=1in]{geometry}

% i like the black square better.
\renewcommand{\qedsymbol}{$\blacksquare$}

\newcommand{\Q}{\mathbb{Q}}
\newcommand{\Z}{\mathbb{Z}}

\newtheorem{theorem}{Theorem}

% let's begin
\begin{document}

\textbf{(Q6)}
\begin{theorem}
    Prove the following statement is true, by proving its contrapositive is true.
    \begin{quote}
        "If $x+y \notin \Q$, then $x \notin \Q$ or $y \notin \Q$."
    \end{quote}
\end{theorem}

\begin{proof}
    The given statement can be expressed mathematically as:
    \[
        x + y \notin \Q \Rightarrow x \notin \Q \text{ or } y \notin \Q
    \]
    The contrapositive of which is:
    \[
        x \in \Q \text{ and } y \in \Q \Rightarrow x + y \in \Q
    \]
    By the definition of rational numbers, we can assign values to $x$ and $y$ as such:
    \begin{quote}
        Let $x = \tfrac{a}{b}, y = \tfrac{c}{d}$, where $a,b,c,d \in \Z$ and $b,d \neq 0$.
    \end{quote}
    Therefore,
    \begin{align*}
        x + y & = \frac{ad}{bd} + \frac{bc}{bd}\\
        & = \frac{ad + bc}{bd}
    \end{align*}
    This holds because $b,d \neq 0$, thus $bd \neq 0$.

    We assume that the set of integers is closed under addition and multiplication,
    thus $ad$, $bc$, and $ad + bc \in \Z$.

    Therefore,$\tfrac{ad + bc}{bd}$ is a rational number, which means the contrapositive is true,
    and thus the original statement.
\end{proof}
\end{document}