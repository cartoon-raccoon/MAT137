\documentclass[12pt, letterpaper]{article}
\setlength{\parindent}{0pt}
\usepackage{amsmath}
\usepackage{amsthm}
\usepackage{amssymb}
\usepackage[a4paper, portrait, margin=1in]{geometry}

% i like the black square better.
\renewcommand{\qedsymbol}{$\blacksquare$}

\newcommand{\Q}{\mathbb{Q}}
\newcommand{\Z}{\mathbb{Z}}

\newtheorem{theorem}{Theorem}
\newtheorem{lemma}{Lemma}

% let's begin
\begin{document}
\textbf{(Q7)}

\textit{(a)} True.
\begin{proof}
    The original statement can be expressed as:
    \[
        \forall p,q \in \Q, p + 2q \in \Q
    \]
    Since $2 = \tfrac{2}{1}$, $2 \in \Q$.

    \begin{align*}
        2q & = \frac{2}{1} \cdot \frac{q}{1}\\
        & = \frac{2q}{1} \Rightarrow 2q \in \Q
    \end{align*}
    Since $\Q$ is closed under addition, $p + 2q \in \Q$.
\end{proof}

\textit{(b)} False.
\begin{proof}
    We can prove this by contradiction.

    First, we assume the following lemma:

    \[
        p \in \Q \setminus \{0\}, q \notin \Q \Rightarrow pq \notin \Q
    \]
    The product of a non-zero rational number and an irrational number
    is always an irrational number.

    The negation of the given statement is:
    \[
        \exists p, q \notin \Q \text{ s.t. } p \neq \pm q \text{ and } pq \notin \Q\\
    \]
    We can assume that $\sqrt{2} \notin \Q$, so we can set $p = \sqrt{2}$ and $q = 2\sqrt{2}$.
    This holds because of the lemma, so $p \neq \pm q$ and $p,q \notin \Q$.

    Thus:
    \begin{align*}
        pq & = 4 \in \Q \Rightarrow pq \in \Q
    \end{align*}

    Which contradicts the given statement.

\end{proof}
\newpage

\textit{(c)} True.
\begin{proof}
    In order to prove the given statement, we have to break it down into its components
    and check that each of them meets the definitions for each operation.

    Thus, we can break down $\tfrac{p^{2} + q}{q}$ into three components:
    \begin{itemize}
        \item The multiplication of $p$ and $p$, to give $p^{2}$.
        \item The addition of $p^{2}$ and $q$, to give $p^{2} + q$.
        \item The division of $p^{2} + q$ over $q$ to, to give $\tfrac{p^{2} + q}{q}$.
    \end{itemize}
    First, we check $p^{2}$.

    $p^{2}$ can be defined as $\tfrac{p}{1} \cdot \tfrac{p}{1}$, from which follows:
    \begin{align*}
        \frac{p}{1} \cdot \frac{p}{1} & = \frac{p^{2}}{1}\\
        & = p^{2}\\
        & \Rightarrow p^{2} \in \Q
    \end{align*}
    Now that we have established that $p^{2}$ is rational, we can check $p^{2} + q$.
    
    By definition of rational numbers,
    \begin{align*}
        p^{2} + q & = \frac{p^{2}}{1} + \frac{q}{1}\\
        & = \frac{p^{2} + q}{1}\\
        & = p^{2} + q
    \end{align*}
    By (a), we can assume $p^{2} + q$ is a rational number.
    Finally we can check the division $p^{2} + q$ by $q$.
    \begin{align*}
        \frac{p^{2} + q}{q} & = \frac{p^{2}}{q} + \frac{q}{q}\\
        & = \frac{p^{2}}{q} + 1
    \end{align*}
    Since 1 is an integer and thus a rational number, we can focus on $\tfrac{p^{2}}{q}$.
    \begin{align*}
        \frac{\tfrac{p^{2}}{1}}{\tfrac{q}{1}} & \in \Q\\
        & = \frac{p^{2}}{q}\\
        & \Rightarrow \frac{p^{2}}{q} \in \Q
    \end{align*}
    This holds because $q \in \Q \setminus \{0\} \Rightarrow q \neq 0$.
    
    Since by (a), the sum of two rational numbers is always rational, the proof is complete.
\end{proof}
\newpage
\textit{(d)} True.

\begin{proof}
    We can prove this by contradiction.

    The negation of the given statement is:
    \[
        \exists p \in \Q \text{ and } 2q \notin \Q \text{ s.t. } pq \in \Q
    \]

    Assuming this is true, we can assume $pq \in \Q$, which implies $2q \cdot p \in \Q$.
    
    Let $p = \tfrac{a}{b}$ where $a,b \in \Z \text{ and } b \neq 0$, and $x \in \Q$. Thus:
    \begin{align*}
        \frac{a}{b} \cdot 2q & = x\\
        \frac{b}{a} \cdot \frac{a}{b} \cdot 2q & = x \cdot \frac{b}{a}\\
        & = 2q\\
        \text{Since } x \cdot \frac{b}{a} \in \Q,2q \in \Q
    \end{align*}

    Which contradicts the initial statement, where $2q \notin \Q$.
\end{proof}

\end{document}