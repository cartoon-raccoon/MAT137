\documentclass[12pt, letterpaper]{article}
\setlength{\parindent}{0pt}
\usepackage{amsmath}
\usepackage{amsthm}
\usepackage{amssymb}

\begin{document}
\textbf{(Q4)}

The contrapositive of a conditional statement $A \Rightarrow B$ is $\neg B \Rightarrow \neg A$.

This is because $A \Rightarrow B$ does not go both ways:

\[
    A \Rightarrow B \neq B \Rightarrow A
\]
Intuitively speaking, $A \Rightarrow B$ means that "If $A$ is true, then $B$ must be true." This does
not necessarily imply that "If $B$ is true, then $A$ must be true", as $B$ being true does not
necessarily mean that $A$ is true.

In order to achieve the logical equivalent of $A \Rightarrow B$, we need to negate both the
condition and outcome in $B \Rightarrow A$:
\[
    A \Rightarrow B = \neg B \Rightarrow \neg A
\]
This clearly makes more sense, as from the earlier intuitive definition, if $B$ is false then
we know that $A$ is false.

To apply this to the given statement:
\begin{quote}
    If it's raining, then it's cloudy.
\end{quote}

The logical equivalent of this statement is not "If it's cloudy, then it's raining".
We know this intuitively: Just because it's cloudy, it doesn't mean there
is rain. In order to achieve logical equivalency, we need to negate both sides:
\begin{quote}
    If it's not cloudy, then it's not raining.
\end{quote}
Which is the given contrapositive of the given statement.


\end{document}