\documentclass[12pt, a4paper]{article}
\setlength{\parindent}{0pt}
\usepackage{amsmath}
\usepackage{amsthm}
\usepackage{amssymb}
\usepackage[a4paper, portrait, margin=1in]{geometry}

% i like the black square better.
\renewcommand{\qedsymbol}{$\blacksquare$}
%\renewcommand{\implies}{\Rightarrow}

\newtheorem{theorem}{Theorem}

% let's begin
\begin{document}

\textbf{(Q2)}
\begin{theorem}
    Let $A,B$ be sets. Prove that if $(A \cap B)^{c} = A^{c}$ then $A \setminus B = \phi$.
\end{theorem}

\begin{proof}
    We can prove this by contradiction.

    We know that $A \setminus B = \phi$ iff $A \setminus B \subseteq \phi$ and $\phi \subseteq A \setminus B$. 
    
    Since $\phi \subseteq A \setminus B$ by its own definition, we only need
    to prove that $A \setminus B \subseteq \phi$.

    For the sake of contradiction, we assume that $A \setminus B \nsubseteq \phi$. 
    This results in the implication:
    \begin{equation}
        \exists x \:\text{s.t.}\: x \in A \text{ and } x \notin B
    \end{equation}

    This implies $x \notin A^{c}$. Assuming $(A \cap B)^{c} = A^{c}$ as given, we can rewrite this as:
    \[
        x \notin \left(A \cap B\right)^{c} \implies x \in \left(A \cap B\right)
    \]
    Which implies that $x \in A$ and $x \in B$. However, this is not possible as it contradicts
    the earlier statement $x \in A \text{ and } x \notin B$.

    Therefore, $A \setminus B \subseteq \phi$, which combined with $\phi \subseteq A \setminus B$,
    allows us to conclude that $A \setminus B = \phi$.

\end{proof}
\end{document}